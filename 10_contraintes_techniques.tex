\section{Contraintes Techniques}

La mise en place d'un fablab induit une série de contraintes techniques, matérielles et humaines. Les deux prochains
chapitres seront dédiés à l'examen de celles-ci.

\subsection{Matériel à disposition}

Nous l'avons déjà évoqué, le fablab doit mettre a disposition de ses membres du matériel pour travailler.

Certains fablab s'orientent parfois vers un secteur d'activité particulier, mais on peut distinguer quelques outils et machines
présents en règle générale.

\subsubsection{Imprimante 3D (phase 0)}

L'imprimante 3D, qui a vu son utilisation exploser ces dernières années, est un grand classique des fablabs. Depuis les
\textit{reprap} (imprimantes auto-réplicatives) bricolées sur un coin de table aux MakerBot calibrées au dixième, on trouve de
tout. Une chose toutefois, la plupart du temps le fablab s'équipe d'une ou deux imprimantes d'excellente qualité et un
\textit{pool} d'imprimantes de qualité correcte mais sans plus pour imprimer le tout venant. En fait, un projet possible pour
le lab est l'impression de repraps avec les imprimantes de bonne qualité pour alimenter le \textit{pool} général.

\subsubsection{Fraiseuse Numérique (phase 1)}

Ce que l'imprimante ne fait pas, la fraiseuse s'en charge. Avec la possibilité d'usiner toute une série de matériaux
avec une excellente précision (meilleure que sur les imprimantes), les fraiseuses sont indispensable au bon
fonctionnement d'un lab. Dans un premier temps, une seule (avec un bon volume de travail) suffira.

\subsubsection{Découpeuse LASER (phase 2)}

Les découpeuses LASER sont des machines de précision, bien documentées et relativement abordables. Leur utilisation
permet la découpe de nombreux matériaux et rapidement.

Une découpeuse LASER est de toute façon à prévoir, même si elle n'arrive que dans un second temps.

\subsubsection{Découpeuse plasma (phase 2)}

La découpeuse plasma (idéalement CNC) permet la découpe de plaques métaliques. Une torche plasma vient fondre le métal
tandis qu'un jet éjecte les goutelettes de métal fondu. La découpeuse plasma a comme principal inconvénient la
production de gaz toxiques. Cet inconvénient peut être annulé en choisissant une machine équipé d'un jet d'eau (qui aura
aussi pour effet de refroidir le métal immédiatement après la découpe).

\subsubsection{Tour Numérique (phase 1)}

Un tour permet de réaliser les pièces cylindriques. Le tour est utile notament pour le filetage de tiges, usinage
d'épaulements, etc...

\subsubsection{Poste de soudure à l'arc (phase 3)}

Lorsque les projets prennent de l'ampleur, il est parfois (souvent) nécessaire de changer le prototype en bois pour un
produit plus robuste. Le poste de soudure à l'arc permet de réaliser des soudures courantes et de faire aboutir une
certaine classe de projets. Cet élément est facultatif. On peut aussi envisager des postes TIG ou MIG pour d'autres
métaux.

\subsubsection{Dégauchisseuse / Combiné (phase 3)}

Un combiné dégauchisseuse/raboteuse/mortaiseuse permet de travailler le bois. C'est un élément important et polyvalent.
Ces machines, si elles donnent d'excellents résultats, demandent une certaine formation.

\subsubsection{Outillage à main (phase 0)}

Il faut ajouter au matériel précédement cité des outils "à main". Voici une liste du matériel utile en au moins un
exemplaire :

\begin{itemize}
	\item meuleuse
	\item ponceuse
	\item défonceuse
\end{itemize}

En sus, il faudra absolument prévoir (au moins):

\begin{itemize}
	\item Deux outils multi-usage type Dremel (puissances différentes) et des jeux d'accessoire compatibles
	\item Deux ou trois perceuses/viseuses à batterie et des jeux de forets (bois, métal, et eventuellement béton)
\end{itemize}

\subsection{Outillage non-électrique}

Beaucoup de projets sont réalisables à la main. Il est important de ne pas négliger le petit-outillage et les
fondamentaux :

\begin{itemize}
	\item scies bois et métal (notament scies à chantourner, scies de précision, scies à onglet et scies égoines)
	\item tournevis : porte-embouts + embouts, tournevis de précision
	\item clés allen, clés à pipes, clés plates
	\item marteaux, masses
	\item ciseaux à bois (plats, gouges, V)
	\item rabots à bois (au moins de petite taille)
	\item limes (plusieurs grains différents)
	\item papier de verre (plusieurs grains différents)
	\item règlets (en 50cm et en 100cm), équerres
	\item pieds à coulisse numériques (x4)
	\item jeux de pinces pour l'électronique
	\item établis
	\item étaux
\end{itemize}

\subsubsection{Matériel pour l'électronique}

La fabrication de montages électroniques nécessite du matériel particulier. Aussi, il faudra acquérir pour chaque "poste"
électrique/électronique :

\begin{itemize}
	\item oscilloscope numérique
	\item oscilloscope analogique (1 pour 2 postes, en récupération)
	\item station de soudure type Weller
	\item alimentation(s) stabilisée (soit une avec 2 sorties, soit 2)
	\item multimètre de labo et multimètre à main
	\item pompe à déssouder
	\item Générateur Basse Fréquence (GBF)
	\item lampe et lampe loupe
\end{itemize}

On peut considérer que dans un premier temps, 4 postes équipés forment un bon équipement.
Ce à quoi s'ajoute (pour les animations ou l'utilisation générale) :

\begin{itemize}
	\item analyseurs de spectre (un avec une gamme megahertz et un en audio)
	\item fers à souder (x10)
	\item loupes binoculaires (x2)
\end{itemize}

\subsubsection{EPI}

Les EPI (Equipements de Protection Individuelle) ne \textit{doivent pas} être négligés.

Même si les utilisateurs sont chargés de les vérifier avant de les utiliser, il faut s'assurer qu'un ensemble de
protections suffisantes seront à disposition. Il faut bien insister auprès des usagers sur le fait que, si le lab n'est
pas en mesure de leur fournir les protections, ils doivent \textit{eux-même} s'en charger.

Dans les équipements nécessaires, on trouvera notament des gants (cuir, étirables, et manutention), des lunettes, les
masques recommandés pour la soudure.

Une trousse à pharmacie complète est plus que souhaitable.

