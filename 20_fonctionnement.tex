\section{Fonctionnement}

Le fablab est un lieu complexe avec beaucoup de gens et de projets qui s'y croisent.

\subsection{Lab manager}

Le \textit{lab manager} est la personne qui est chargée de faire vivre le lab. Elle accueille les \textit{makers}, les forme à
l'utilisation des différentes machines et les aide au besoin.

Elle doit aussi avoir une bonne base technique et la maîtrise d'une vaste majorité du matériel du lab. Il est en
charge de la maintenance générale des équipements, de la tenue du lab, de la communication autour de celui-ci.

Enfin, le \textit{lab manager} est chargé des partenariats externes et de la gestion du stock.

Afin de répondre au mieux au besoin, le manager devra notament :

\begin{itemize}
	\item avoir des bases en électricité et en électronique
	\item savoir utiliser les appareils de mesure et de fabrication de la zone électrique (oscilloscope, multimètre, station de soudure, etc...)
	\item connaître les règles de sécurité liées à l'utilisation du matériel à disposition
	\item savoir utiliser l'imprimante actuellement à disposition
	\item être capable de se former au matériel de la zone mécanique (avant utilisation)
	\item connaître les règles de sécurité liées aux machines-outils
	\item savoir identifier les problèmes pouvant apparaître lors de la production de pièces mécaniques (dévers, contraintes, etc...)
	\item savoir utiliser les moyens métrologiques à disposition en zone mécanique
\end{itemize}

Dans le cadre des activités du lab, le manager devra aussi :

\begin{itemize}
	\item faire preuve pédagogie pour expliquer le fonctionnement des machines
	\item savoir gérer les partenariats éventuels
	\item savoir intéragir avec une communauté et communiquer sur les actions
\end{itemize}

Une connaissance des méthodes industrielles et une expérience des machines à disposition en zone mécanique sont autant
d'avantages.

\subsection{Magasin}

L'idée générale est de fournir 2 types d'adhésions : une réservée aux porteurs de projet et une aux entreprises.

\subsubsection{Accès aux porteurs de projets}

Les porteurs de projets arrivent avec un projet et repartent avec la réalisation. Les systèmes développés dans le lab
leur appartiennent (propriété intellectuelle et physique).

Idéalement, il faudrait que l'adhésion des porteurs de projets comporte :

\begin{itemize}
	\item le prix d'accès au lab (permettant de couvrir les frais de fonctionnement)
	\item le prix de la matière première pour l'impression
	\item un accès à un peu de visserie
	\item un accès aux établis et imprimantes (pour des impressions de moins d'une heure par exemple)
\end{itemize}

Une cinquantaine d'euros par an semble un bon compromis dans un premier temps.

\subsubsection{Accès aux machines pour les porteurs de projets}

Toutes les machines ont un coût de fonctionnement, d'entretien et une consommation en eau/électricité.

Un tarif horaire d'accès aux machines permettra de rendre exploitable le lab (c'est d'ailleurs ce qui est proposé dans
d'autres lieux).

L'entretien est par ailleurs dévolu aux utilisateurs du lab : cela comprend notament le ménage, mais aussi la réparation
des machines, leur nettoyage, la réparation du matériel, etc...

\subsubsection{L'accès aux entreprises}

Comme l'ont fait certains labs, des plages horaires peuvent être réservés aux entreprises. Plusieurs solutions sont
envisageables, l'une d'entre elles semble intéressante : une entreprise adhère au lab et se voit réserver le lab entier une journée
(9h-18h) par mois. On peut aussi envisager d'ajouter au forfait un accès aux journées "entreprise" (sans garantie
d'avoir le lab pour elle seule, mais avec l'assurance que seules des entreprises seront présentes).

Une attention particulière devra être portée à la communication auprès des entreprises : le fablab n'est \textbf{pas} un
service de R\&D et le \textit{lab manager} n'est pas un ingénieur à disposition.

\subsubsection{Accès au magasin}


Lors d'un projet, on a besoin de matière première et souvent d'un peu de quincaillerie.

Comme dit plus ahut, la mise en place d'un magasin est une bonne idée. Celui-ci devrait comprendre :

\begin{itemize}
	\item des matières premières pour la fabrication (filament, plaques d'aluminium, de bois, planches)
	\item de la quincaillerie générale (visserie, clous, etc...)
	\item des composants électroniques de base (passifs, résistances, condensateurs, transistors, convertisseurs, etc...)
	\item du consommable (tresse à dessouder, étain, scotch, etc...)
\end{itemize}

L'intérêt du magasin est de fournir aux \textit{makers} présents tous les matériaux dont ils ont besoin pour mener à bien leurs
projets. Si le fablab est bien géré, cela permettra aussi de bénéficier de tarifs de gros sur les composants et le
consommable.

\subsection{Horaires}

Afin de garantir une efficacité maximale, le fablab devra être ouvert le plus souvent possible. Un \textit{lab manager} est
nécessaire pour assurer l'ouverture à destination des entreprises et la gestion du lab en lui-même. Les horaires du
manager ne seront toutefois pas extensibles et il faudra probablement trouver un moyen de faire vivre le lab aussi en
soirée pour les porteurs de projet qui ne peuvent se libérer dans la journée (mise à disposition pour des associations
en échange du fait qu'elles animent le lieu).
