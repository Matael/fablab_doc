\section{Lieux}

Dans le meilleur des cas (pour une utilisation électronique et mécanique), on peut distinguer 7 zones. Pour les deux
premières (zones mécanique et électronique) au moins, un sol avec revêtement résiné est essentiel. De même, des
évacuations d'eau au sol sont éventuellement à prévoir.

\subsection{Zone mécanique}

La zone mécanique est celle où sont les machines outils, celle où on travaille sur les pièces elles-mêmes. Elle doit
être spacieuse et aérée (sécurité). Il est important de prévoir un éclairage efficace sur cette zone pour que
l'utilisation des machines se fasse dans de bonnes conditions.

Les postes d'usinage nécessitent souvent un ordinateur pour le contrôle ; aussi  des accès réseau (RJ45) sont important
de même que des prises électriques. Certaines machines peuvent nécessiter des prises de puissance triphasées.

\subsection{Zone électronique}

Equipée de paillasses de labo pour travailler cette zone doit etre bien aérée (vapeurs de soudure). Les paillasses
doivent disposer d'un accès réseau et de nombreuses prises électriques.

\subsection{Zone conception}

Une ou deux grandes tables permettent de travailler ensemble sur les différents projets. Quelques ordinateurs sont
souhaitables mais surtout un accès wifi pour les PC des makers est important.

Un endroit aéré et clair est plus agréable pour travailler par ailleurs, les activités du lab pourrait dégrader la
table, aussi il est préférable de prévoir un plateau de bois épais et des pieds à viser plutot qu'une table avec un
revêtement plastique. Les tables en métal sont exclues (electricité statique).

\subsection{Zone détente}

Le fablab possède une dimension sociale importante. Il est important de prévoir une zone où prendre un café, manger un
peu, discuter. Dans certains labs, ça se résume à une table et quelques chaises pliantes, un peu d'espace pas trop loin
de la zone de conception et un meuble pour ranger le consommable (café et assimilés).

\subsection{Zone administration}

Un bureau un tant soit peu fermé pour le \textit{lab manager} est quelque chose qu'on oublie parfois mais qui est important. En
fait, pas besoin d'avoir forcément très grand ou très fermé mais juste un coin avec une cloison et un bureau suffit la
plupart du temps.

\subsection{Stockage}

Un endroit (vérouillable) pour stocker les matériaux et les composants du magasin (qui sera évoqué au chapitre suivant)
est essentiel. La fabrication de pièces sous-entend en effet d'avoir des matières premières et un endroit pour les
stocker. A prévoir donc, quelques étagères, et un peu de place au sol pour ça.

\subsection{Espace extérieur}

Certaines activités ne peuvent se faire en intérieur (certaines peintures, montage de structures, etc...), il est donc
intéressant de prévoir une dalle béton à l'extérieur pour pouvoir y travailler. En fait, les deux intérêts principaux de
la dalle béton sont qu'elle fourni un espace de travail plat et de niveau et qu'elle évite la poussière au sol
(nettoyable).

\subsection{Accès réseau}

Pour éviter les problème de maintenance et l'intervention de techniciens, l'accès réseau est assuré par l'association
résidente. Les membres du lab se chargent de la souscription, de la maintenance et de la gestion.

\subsection{Circuits Electriques}

L'installation électrique doit pendre en compte le découplage des différentes zones.

La zone mécanique peut nécessiter des prises de puissance. Elles devront alors être sur des circuits séparés.

Les postes de la zone électrique devront être sur des circuits séparés de sorte qu'une erreur sur l'un ne provoquera pas
l'arrêt de toute la zone ou pire de tout le lab.

L'arrêt soudain d'une machine outil peut être dangereux pour l'opérateur.

