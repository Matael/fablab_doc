\section{Lancement du lab}

Nous allons maintenant examiner les premiers mois d'existence du lab.

\subsection{Affluence}

Dans un premier temps, le lab ne sera pas très fréquenter : en effet la mise en place d'une communauté est quelque chose
d'assez long. Il est important d'assurer une présence sur tous les évènements et toutes les manifestations qui
pourraient amener plus de personnes à utiliser le lieux.

Une faible affluence dans un premier temps n'est absolument pas un point alarmant. Un bon indicateur de la qualité du
lab peut être d'observer le taux de rebond : si trop de personnes ne viennent qu'une fois sans jamais recontacter alors
il faut envisager de réadapter la communication.

Sous peu de temps, des associations partenaires commenceront à demander l'accès au lab pour y organiser des évènements
ou bénéficier d'un environnement de travail. Il faut systématiquement accéder à leurs demandes (en composant avec
tous les différents acteurs bien évidement).

\subsection{Communication Externe}

La communication autour du lab doit être massive. En plus de la présence aux évènements, il faut alimenter un compte
twitter et un flux photo (type flickr) pour donner envie aux usagers potentiels et leur offrir un moyen de se tenir au
courant.

\subsection{Communication interne}

La mise en place d'une mailing list interne n'est pas optionnelle. La présence sur les mailing lists d'autres
associations non plus.

En plus de ça, un wiki pourra avantageusement permettre d'aggréger les connaissances et les conseils. La dimension
participative du wiki permettra a tous d'y trouver leur compte.

Enfin, la mise en place d'un canal IRC pourrait amener plus d'interaction. Il n'est pas absolument nécessaire d'en créer un
mais peut être d'utiliser dans un premier temps celui d'une association partenaire.

\subsection{Opensource, Projets libres}

Le \textit{lab manager} devra être dans un premier temps présent au sein des communautés \textit{makers} en se faisant connaître sur
les forums. C'est un excellent moyen de faire connaître le lab.

La participation à des projets opensource est clairement un plus.

\subsection{Projets internes}

Il faut enfin créer et faire vivre quelques projets en interne pour avoir de quoi communiquer. Ces projets peuvent être
de tout type mais pas forcément très techniques. Dans un premier temps, plus ils sont visuels et facile à comprendre et
plus ils parleront au public.

\subsection{Formations}

Un dernier moyen d'attirer de nouveaux usagers est de proposer des formations sur le matériel du lab. Ces formations
peuvent être assurées par des associations tierces mais il est important que le fablab s'y fasse connaître.

La qualité des formations est importante pour ne pas détruire le projet, mais le niveau de celles-ci n'a pas forcément
besoin d'être extrèmement élevé.

Autant que possible, il faudra que les formations soient gratuites (quitte à faire payer le matériel ou à en vendre).

Il est possible de distinguer aussi les formations à destination du public et des porteurs de projet de celles à
destination des entreprises. En effet, les attentes ne sont pas les mêmes et le niveau requis non plus.

\subsection{Partenariats}

Un fablab ne peut que profiter de partenariats. Les magasins de bricolages, fabricants de matériel, etc... sont des
cibles à privilégier. Ils disposent en effet de matériel à prêter/donner, de matières premières intéressantes et
pourraient bénéficier d'une animation au sein de leur structure (fablab temporaire, etc...).

Les écoles sont aussi des partenaires intéressants : elles peuvent mettre à disposition des salles/du matériel
(ponctuellement), profiter du lab pour animer leurs projets ou réaliser des outils personnalisés et ont tout intérêt à
montrer qu'elles sont partie prenante dans ce type de projet.

