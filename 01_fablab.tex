\section{Un fablab ?}

Fablab et hackerspace ont la même base. Ils encouragent le partage de connaissances, l'entraide, le travail en commun.
L'objectif de ces structures est de fournir à leurs membres un lieu et du matériel pour se former, travailler et faire
vivre leurs projets. Toute la force du mouvement *maker* vient de sa capacité à attirer beaucoup de profils différents,
créant ainsi une communauté soudée et capable.

Là où le hackerspace cherche la compréhension des systèmes par leur démontage et leur modification pour qu'ils
répondent parfaitement au besoin, le fablab s'oriente (comme l'indique son nom *fabrication laboratory*) vers la
création.

Le fablab est un lieu et une structure qui cherche à fournir à ses membres des moyens de production et de prototypage.

Le fablab a aussi un deuxième objectif : celui de fédérer une communauté autour du projet et de la faire vivre.

\section{Intérêt}

Un fablab est un lieu de créativité dynamique. C'est, pour un  territoire, un moyen d'attirer une population
jeune et pleine d'idées, potentiellement créatrice d'entreprises et de toute façon actrice de la vie locale.

C'est aussi un moyen de s'inscrire dans une dynamique participative et de mettre en avant la capacité du territoire à
encourager les initiatives innovantes et l'économie collaborative.

Enfin, le fablab est un moyen de créer des entreprises (*via* les projets qui s'y développent) et de fournir aux
entreprises déjà implantées un endroit où tester leurs idées hors du cadre parfois lent et fermé des départements de R\&D.
