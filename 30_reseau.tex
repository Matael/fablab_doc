\section{Réseau}

Une des forces d'un fablab est la mise en réseau d'acteurs locaux.
Plusieurs actions sont donc à mener dans ce sens.

\subsection{Collaboration avec d'autres associations}

De nombreuses associations déjà implantées poursuivent des buts similaires en ciblant un public particulier.
Les associations d'éducation populaire (Petits Débrouillards, Francas, etc...) sont des partenaires privilégiés car ils
disposent déjà de matériel et de locaux.

Le fablab a vocation à accueillir des évènements organisés par d'autres et à leur permettre d'améliorer leurs
prestations. En échange, le fablab se fait connaître auprès des communautés existantes et, à terme, tend à devenir un
endroit où les acteurs viennent travailler ensemble.

Cette dimension communautaire sert aussi les entreprises liées au fablab : en brassant les acteurs et les connaissances
au sein du fablab, on accélère son développement. Ainsi, plus il aura d'usagers et plus le fablab sera intéressant. De
nouveaux usagers amènent aussi de nouvelles manières de travailler et de voir les choses et ainsi, ils contribuent à
l'élaboration de solutions plus performantes (même indirectement).

Dans cette optique, le poste de \textit{lab manager} prend tout son sens : il est en effet le lien entre tout les acteurs et
celui qui sera chargé de chercher des partenaires éventuels. De même, \textit{via} son rôle de formation, il est un moyen de
propager les connaissances.

Enfin, dans cette optique de mise en réseau, la création d'une fédération est envisagée. Ce montage associatif a deux
niveaux aurait l'intérêt de fournir un cadre juridique prêt à fonctionner pour le partage de locaux et de matériel entre
les acteurs locaux. Pour la CCI, c'est un moyen de valoriser son action dans la création d'une dynamique communautaire
sur le territoire.

\subsection{Collaboration avec les écoles}

Le département compte un certain nombre d'écoles proposant des formations techniques (en conception ou fabrication). Il
est primordial que le lab se fasse connaître auprès d'elles et de leurs étudiants. Si les écoles fournissent une base
théorique (parfois pratique) intéressante, le fablab permet d'héberger les projets peut être moins scolaires mais tout
aussi important dans la formation.

Parmi les partenaires potentiels identifiés, on retrouve l'ISMANS, l'ITEMM, l'ESBAM mais aussi l'ENSIM ou l'UNAM
(Université du Maine). Une partenariat avec le projet TechnoCampus est aussi envisageable.


\subsection{Quand les labs pullulent}

Enfin, il faut faire grossir l'idée.

Cette dernière partie de la recette est loin d'être la plus simple mais c'est elle, en définitive, qui fera la
différence entre le succès et l'échec du projet à un niveau plus global (à l'échelle du département).

Tous les \textit{makers} et autres usagers potentiels ne sont pas concentrés au Mans. Il faut que le fablab (via son
\textit{lab manager}
et ses usagers) fasse vivre l'idée du lab le plus loin possible. Cette action peut s'envisager de plusieurs manières :

\begin{itemize}
	\item la promotion des nouveaux usages \textit{via} des hackathons/bidouillo-thons et fab-o-thons de toute sorte
	\item la présence sur les évènements autour du numérique ou de la culture alternative
	\item l'information populaire, la formation des curieux aux technologies nouvelles
	\item la création de labs temporaires dans divers endroits du département (sur des temps courts, entre 6 et 48h, en partenariat avec les acteurs locaux dans le cadre de Sarthe Numérique par exemple)
	\item la présence sur les réseaux sociaux, la diffusion d'information
\end{itemize}

Par toutes ces actions, le lab se fera (re)connaître et pourra alors devenir un acteur majeur de la vie sociale et
économique locale.
